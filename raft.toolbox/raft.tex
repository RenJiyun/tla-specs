\batchmode %% Suppresses most terminal output.
\documentclass{article}
\usepackage{color}
\definecolor{boxshade}{gray}{0.85}
\setlength{\textwidth}{360pt}
\setlength{\textheight}{541pt}
\usepackage{latexsym}
\usepackage{ifthen}
% \usepackage{color}
%%%%%%%%%%%%%%%%%%%%%%%%%%%%%%%%%%%%%%%%%%%%%%%%%%%%%%%%%%%%%%%%%%%%%%%%%%%%%
% SWITCHES                                                                  %
%%%%%%%%%%%%%%%%%%%%%%%%%%%%%%%%%%%%%%%%%%%%%%%%%%%%%%%%%%%%%%%%%%%%%%%%%%%%%
\newboolean{shading} 
\setboolean{shading}{false}
\makeatletter
 %% this is needed only when inserted into the file, not when
 %% used as a package file.
%%%%%%%%%%%%%%%%%%%%%%%%%%%%%%%%%%%%%%%%%%%%%%%%%%%%%%%%%%%%%%%%%%%%%%%%%%%%%
%                                                                           %
% DEFINITIONS OF SYMBOL-PRODUCING COMMANDS                                  %
%                                                                           %
%    TLA+      LaTeX                                                        %
%    symbol    command                                                      %
%    ------    -------                                                      %
%    =>        \implies                                                     %
%    <:        \ltcolon                                                     %
%    :>        \colongt                                                     %
%    ==        \defeq                                                       %
%    ..        \dotdot                                                      %
%    ::        \coloncolon                                                  %
%    =|        \eqdash                                                      %
%    ++        \pp                                                          %
%    --        \mm                                                          %
%    **        \stst                                                        %
%    //        \slsl                                                        %
%    ^         \ct                                                          %
%    \A        \A                                                           %
%    \E        \E                                                           %
%    \AA       \AA                                                          %
%    \EE       \EE                                                          %
%%%%%%%%%%%%%%%%%%%%%%%%%%%%%%%%%%%%%%%%%%%%%%%%%%%%%%%%%%%%%%%%%%%%%%%%%%%%%
\newlength{\symlength}
\newcommand{\implies}{\Rightarrow}
\newcommand{\ltcolon}{\mathrel{<\!\!\mbox{:}}}
\newcommand{\colongt}{\mathrel{\!\mbox{:}\!\!>}}
\newcommand{\defeq}{\;\mathrel{\smash   %% keep this symbol from being too tall
    {{\stackrel{\scriptscriptstyle\Delta}{=}}}}\;}
\newcommand{\dotdot}{\mathrel{\ldotp\ldotp}}
\newcommand{\coloncolon}{\mathrel{::\;}}
\newcommand{\eqdash}{\mathrel = \joinrel \hspace{-.28em}|}
\newcommand{\pp}{\mathbin{++}}
\newcommand{\mm}{\mathbin{--}}
\newcommand{\stst}{*\!*}
\newcommand{\slsl}{/\!/}
\newcommand{\ct}{\hat{\hspace{.4em}}}
\newcommand{\A}{\forall}
\newcommand{\E}{\exists}
\renewcommand{\AA}{\makebox{$\raisebox{.05em}{\makebox[0pt][l]{%
   $\forall\hspace{-.517em}\forall\hspace{-.517em}\forall$}}%
   \forall\hspace{-.517em}\forall \hspace{-.517em}\forall\,$}}
\newcommand{\EE}{\makebox{$\raisebox{.05em}{\makebox[0pt][l]{%
   $\exists\hspace{-.517em}\exists\hspace{-.517em}\exists$}}%
   \exists\hspace{-.517em}\exists\hspace{-.517em}\exists\,$}}
\newcommand{\whileop}{\.{\stackrel
  {\mbox{\raisebox{-.3em}[0pt][0pt]{$\scriptscriptstyle+\;\,$}}}%
  {-\hspace{-.16em}\triangleright}}}

% Commands are defined to produce the upper-case keywords.
% Note that some have space after them.
\newcommand{\ASSUME}{\textsc{assume }}
\newcommand{\ASSUMPTION}{\textsc{assumption }}
\newcommand{\AXIOM}{\textsc{axiom }}
\newcommand{\BOOLEAN}{\textsc{boolean }}
\newcommand{\CASE}{\textsc{case }}
\newcommand{\CONSTANT}{\textsc{constant }}
\newcommand{\CONSTANTS}{\textsc{constants }}
\newcommand{\ELSE}{\settowidth{\symlength}{\THEN}%
   \makebox[\symlength][l]{\textsc{ else}}}
\newcommand{\EXCEPT}{\textsc{ except }}
\newcommand{\EXTENDS}{\textsc{extends }}
\newcommand{\FALSE}{\textsc{false}}
\newcommand{\IF}{\textsc{if }}
\newcommand{\IN}{\settowidth{\symlength}{\LET}%
   \makebox[\symlength][l]{\textsc{in}}}
\newcommand{\INSTANCE}{\textsc{instance }}
\newcommand{\LET}{\textsc{let }}
\newcommand{\LOCAL}{\textsc{local }}
\newcommand{\MODULE}{\textsc{module }}
\newcommand{\OTHER}{\textsc{other }}
\newcommand{\STRING}{\textsc{string}}
\newcommand{\THEN}{\textsc{ then }}
\newcommand{\THEOREM}{\textsc{theorem }}
\newcommand{\LEMMA}{\textsc{lemma }}
\newcommand{\PROPOSITION}{\textsc{proposition }}
\newcommand{\COROLLARY}{\textsc{corollary }}
\newcommand{\TRUE}{\textsc{true}}
\newcommand{\VARIABLE}{\textsc{variable }}
\newcommand{\VARIABLES}{\textsc{variables }}
\newcommand{\WITH}{\textsc{ with }}
\newcommand{\WF}{\textrm{WF}}
\newcommand{\SF}{\textrm{SF}}
\newcommand{\CHOOSE}{\textsc{choose }}
\newcommand{\ENABLED}{\textsc{enabled }}
\newcommand{\UNCHANGED}{\textsc{unchanged }}
\newcommand{\SUBSET}{\textsc{subset }}
\newcommand{\UNION}{\textsc{union }}
\newcommand{\DOMAIN}{\textsc{domain }}
% Added for tla2tex
\newcommand{\BY}{\textsc{by }}
\newcommand{\OBVIOUS}{\textsc{obvious }}
\newcommand{\HAVE}{\textsc{have }}
\newcommand{\QED}{\textsc{qed }}
\newcommand{\TAKE}{\textsc{take }}
\newcommand{\DEF}{\textsc{ def }}
\newcommand{\HIDE}{\textsc{hide }}
\newcommand{\RECURSIVE}{\textsc{recursive }}
\newcommand{\USE}{\textsc{use }}
\newcommand{\DEFINE}{\textsc{define }}
\newcommand{\PROOF}{\textsc{proof }}
\newcommand{\WITNESS}{\textsc{witness }}
\newcommand{\PICK}{\textsc{pick }}
\newcommand{\DEFS}{\textsc{defs }}
\newcommand{\PROVE}{\settowidth{\symlength}{\ASSUME}%
   \makebox[\symlength][l]{\textsc{prove}}\@s{-4.1}}%
  %% The \@s{-4.1) is a kludge added on 24 Oct 2009 [happy birthday, Ellen]
  %% so the correct alignment occurs if the user types
  %%   ASSUME X
  %%   PROVE  Y
  %% because it cancels the extra 4.1 pts added because of the 
  %% extra space after the PROVE.  This seems to works OK.
  %% However, the 4.1 equals Parameters.LaTeXLeftSpace(1) and
  %% should be changed if that method ever changes.
\newcommand{\SUFFICES}{\textsc{suffices }}
\newcommand{\NEW}{\textsc{new }}
\newcommand{\LAMBDA}{\textsc{lambda }}
\newcommand{\STATE}{\textsc{state }}
\newcommand{\ACTION}{\textsc{action }}
\newcommand{\TEMPORAL}{\textsc{temporal }}
\newcommand{\ONLY}{\textsc{only }}              %% added by LL on 2 Oct 2009
\newcommand{\OMITTED}{\textsc{omitted }}        %% added by LL on 31 Oct 2009
\newcommand{\@pfstepnum}[2]{\ensuremath{\langle#1\rangle}\textrm{#2}}
\newcommand{\bang}{\@s{1}\mbox{\small !}\@s{1}}
%% We should format || differently in PlusCal code than in TLA+ formulas.
\newcommand{\p@barbar}{\ifpcalsymbols
   \,\,\rule[-.25em]{.075em}{1em}\hspace*{.2em}\rule[-.25em]{.075em}{1em}\,\,%
   \else \,||\,\fi}
%% PlusCal keywords
\newcommand{\p@fair}{\textbf{fair }}
\newcommand{\p@semicolon}{\textbf{\,; }}
\newcommand{\p@algorithm}{\textbf{algorithm }}
\newcommand{\p@mmfair}{\textbf{-{}-fair }}
\newcommand{\p@mmalgorithm}{\textbf{-{}-algorithm }}
\newcommand{\p@assert}{\textbf{assert }}
\newcommand{\p@await}{\textbf{await }}
\newcommand{\p@begin}{\textbf{begin }}
\newcommand{\p@end}{\textbf{end }}
\newcommand{\p@call}{\textbf{call }}
\newcommand{\p@define}{\textbf{define }}
\newcommand{\p@do}{\textbf{ do }}
\newcommand{\p@either}{\textbf{either }}
\newcommand{\p@or}{\textbf{or }}
\newcommand{\p@goto}{\textbf{goto }}
\newcommand{\p@if}{\textbf{if }}
\newcommand{\p@then}{\,\,\textbf{then }}
\newcommand{\p@else}{\ifcsyntax \textbf{else } \else \,\,\textbf{else }\fi}
\newcommand{\p@elsif}{\,\,\textbf{elsif }}
\newcommand{\p@macro}{\textbf{macro }}
\newcommand{\p@print}{\textbf{print }}
\newcommand{\p@procedure}{\textbf{procedure }}
\newcommand{\p@process}{\textbf{process }}
\newcommand{\p@return}{\textbf{return}}
\newcommand{\p@skip}{\textbf{skip}}
\newcommand{\p@variable}{\textbf{variable }}
\newcommand{\p@variables}{\textbf{variables }}
\newcommand{\p@while}{\textbf{while }}
\newcommand{\p@when}{\textbf{when }}
\newcommand{\p@with}{\textbf{with }}
\newcommand{\p@lparen}{\textbf{(\,\,}}
\newcommand{\p@rparen}{\textbf{\,\,) }}   
\newcommand{\p@lbrace}{\textbf{\{\,\,}}   
\newcommand{\p@rbrace}{\textbf{\,\,\} }}

%%%%%%%%%%%%%%%%%%%%%%%%%%%%%%%%%%%%%%%%%%%%%%%%%%%%%%%%%
% REDEFINE STANDARD COMMANDS TO MAKE THEM FORMAT BETTER %
%                                                       %
% We redefine \in and \notin                            %
%%%%%%%%%%%%%%%%%%%%%%%%%%%%%%%%%%%%%%%%%%%%%%%%%%%%%%%%%
\renewcommand{\_}{\rule{.4em}{.06em}\hspace{.05em}}
\newlength{\equalswidth}
\let\oldin=\in
\let\oldnotin=\notin
\renewcommand{\in}{%
   {\settowidth{\equalswidth}{$\.{=}$}\makebox[\equalswidth][c]{$\oldin$}}}
\renewcommand{\notin}{%
   {\settowidth{\equalswidth}{$\.{=}$}\makebox[\equalswidth]{$\oldnotin$}}}


%%%%%%%%%%%%%%%%%%%%%%%%%%%%%%%%%%%%%%%%%%%%%%%%%%%%
%                                                  %
% HORIZONTAL BARS:                                 %
%                                                  %
%   \moduleLeftDash    |~~~~~~~~~~                 %
%   \moduleRightDash    ~~~~~~~~~~|                %
%   \midbar            |----------|                %
%   \bottombar         |__________|                %
%%%%%%%%%%%%%%%%%%%%%%%%%%%%%%%%%%%%%%%%%%%%%%%%%%%%
\newlength{\charwidth}\settowidth{\charwidth}{{\small\tt M}}
\newlength{\boxrulewd}\setlength{\boxrulewd}{.4pt}
\newlength{\boxlineht}\setlength{\boxlineht}{.5\baselineskip}
\newcommand{\boxsep}{\charwidth}
\newlength{\boxruleht}\setlength{\boxruleht}{.5ex}
\newlength{\boxruledp}\setlength{\boxruledp}{-\boxruleht}
\addtolength{\boxruledp}{\boxrulewd}
\newcommand{\boxrule}{\leaders\hrule height \boxruleht depth \boxruledp
                      \hfill\mbox{}}
\newcommand{\@computerule}{%
  \setlength{\boxruleht}{.5ex}%
  \setlength{\boxruledp}{-\boxruleht}%
  \addtolength{\boxruledp}{\boxrulewd}}

\newcommand{\bottombar}{\hspace{-\boxsep}%
  \raisebox{-\boxrulewd}[0pt][0pt]{\rule[.5ex]{\boxrulewd}{\boxlineht}}%
  \boxrule
  \raisebox{-\boxrulewd}[0pt][0pt]{%
      \rule[.5ex]{\boxrulewd}{\boxlineht}}\hspace{-\boxsep}\vspace{0pt}}

\newcommand{\moduleLeftDash}%
   {\hspace*{-\boxsep}%
     \raisebox{-\boxlineht}[0pt][0pt]{\rule[.5ex]{\boxrulewd
               }{\boxlineht}}%
    \boxrule\hspace*{.4em }}

\newcommand{\moduleRightDash}%
    {\hspace*{.4em}\boxrule
    \raisebox{-\boxlineht}[0pt][0pt]{\rule[.5ex]{\boxrulewd
               }{\boxlineht}}\hspace{-\boxsep}}%\vspace{.2em}

\newcommand{\midbar}{\hspace{-\boxsep}\raisebox{-.5\boxlineht}[0pt][0pt]{%
   \rule[.5ex]{\boxrulewd}{\boxlineht}}\boxrule\raisebox{-.5\boxlineht%
   }[0pt][0pt]{\rule[.5ex]{\boxrulewd}{\boxlineht}}\hspace{-\boxsep}}

%%%%%%%%%%%%%%%%%%%%%%%%%%%%%%%%%%%%%%%%%%%%%%%%%%%%%%%%%%%%%%%%%%%%%%%%%%%%%
% FORMATING COMMANDS                                                        %
%%%%%%%%%%%%%%%%%%%%%%%%%%%%%%%%%%%%%%%%%%%%%%%%%%%%%%%%%%%%%%%%%%%%%%%%%%%%%

%%%%%%%%%%%%%%%%%%%%%%%%%%%%%%%%%%%%%%%%%%%%%%%%%%%%%%%%%%%%%%%%%%%%%%%%%%%%%
% PLUSCAL SHADING                                                           %
%%%%%%%%%%%%%%%%%%%%%%%%%%%%%%%%%%%%%%%%%%%%%%%%%%%%%%%%%%%%%%%%%%%%%%%%%%%%%

% The TeX pcalshading switch is set on to cause PlusCal shading to be
% performed.  This changes the behavior of the following commands and
% environments to cause full-width shading to be performed on all lines.
% 
%   \tstrut \@x cpar mcom \@pvspace
% 
% The TeX pcalsymbols switch is turned on when typesetting a PlusCal algorithm,
% whether or not shading is being performed.  It causes symbols (other than
% parentheses and braces and PlusCal-only keywords) that should be typeset
% differently depending on whether they are in an algorithm to be typeset
% appropriately.  Currently, the only such symbol is "||".
%
% The TeX csyntax switch is turned on when typesetting a PlusCal algorithm in
% c-syntax.  This allows symbols to be format differently in the two syntaxes.
% The "else" keyword is the only one that is.

\newif\ifpcalshading \pcalshadingfalse
\newif\ifpcalsymbols \pcalsymbolsfalse
\newif\ifcsyntax     \csyntaxtrue

% The \@pvspace command makes a vertical space.  It uses \vspace
% except with \ifpcalshading, in which case it sets \pvcalvspace
% and the space is added by a following \@x command.
%
\newlength{\pcalvspace}\setlength{\pcalvspace}{0pt}%
\newcommand{\@pvspace}[1]{%
  \ifpcalshading
     \par\global\setlength{\pcalvspace}{#1}%
  \else
     \par\vspace{#1}%
  \fi
}

% The lcom environment was changed to set \lcomindent equal to
% the indentation it produces.  This length is used by the
% cpar environment to make shading extend for the full width
% of the line.  This assumes that lcom environments are not
% nested.  I hope TLATeX does not nest them.
%
\newlength{\lcomindent}%
\setlength{\lcomindent}{0pt}%

%\tstrut: A strut to produce inter-paragraph space in a comment.
%\rstrut: A strut to extend the bottom of a one-line comment so
%         there's no break in the shading between comments on 
%         successive lines.
\newcommand\tstrut%
  {\raisebox{\vshadelen}{\raisebox{-.25em}{\rule{0pt}{1.15em}}}%
   \global\setlength{\vshadelen}{0pt}}
\newcommand\rstrut{\raisebox{-.25em}{\rule{0pt}{1.15em}}%
 \global\setlength{\vshadelen}{0pt}}


% \.{op} formats operator op in math mode with empty boxes on either side.
% Used because TeX otherwise vary the amount of space it leaves around op.
\renewcommand{\.}[1]{\ensuremath{\mbox{}#1\mbox{}}}

% \@s{n} produces an n-point space
\newcommand{\@s}[1]{\hspace{#1pt}}           

% \@x{txt} starts a specification line in the beginning with txt
% in the final LaTeX source.
\newlength{\@xlen}
\newcommand\xtstrut%
  {\setlength{\@xlen}{1.05em}%
   \addtolength{\@xlen}{\pcalvspace}%
    \raisebox{\vshadelen}{\raisebox{-.25em}{\rule{0pt}{\@xlen}}}%
   \global\setlength{\vshadelen}{0pt}%
   \global\setlength{\pcalvspace}{0pt}}

\newcommand{\@x}[1]{\par
  \ifpcalshading
  \makebox[0pt][l]{\shadebox{\xtstrut\hspace*{\textwidth}}}%
  \fi
  \mbox{$\mbox{}#1\mbox{}$}}  

% \@xx{txt} continues a specification line with the text txt.
\newcommand{\@xx}[1]{\mbox{$\mbox{}#1\mbox{}$}}  

% \@y{cmt} produces a one-line comment.
\newcommand{\@y}[1]{\mbox{\footnotesize\hspace{.65em}%
  \ifthenelse{\boolean{shading}}{%
      \shadebox{#1\hspace{-\the\lastskip}\rstrut}}%
               {#1\hspace{-\the\lastskip}\rstrut}}}

% \@z{cmt} produces a zero-width one-line comment.
\newcommand{\@z}[1]{\makebox[0pt][l]{\footnotesize
  \ifthenelse{\boolean{shading}}{%
      \shadebox{#1\hspace{-\the\lastskip}\rstrut}}%
               {#1\hspace{-\the\lastskip}\rstrut}}}


% \@w{str} produces the TLA+ string "str".
\newcommand{\@w}[1]{\textsf{``{#1}''}}             


%%%%%%%%%%%%%%%%%%%%%%%%%%%%%%%%%%%%%%%%%%%%%%%%%%%%%%%%%%%%%%%%%%%%%%%%%%%%%
% SHADING                                                                   %
%%%%%%%%%%%%%%%%%%%%%%%%%%%%%%%%%%%%%%%%%%%%%%%%%%%%%%%%%%%%%%%%%%%%%%%%%%%%%
\def\graymargin{1}
  % The number of points of margin in the shaded box.

% \definecolor{boxshade}{gray}{.85}
% Defines the darkness of the shading: 1 = white, 0 = black
% Added by TLATeX only if needed.

% \shadebox{txt} puts txt in a shaded box.
\newlength{\templena}
\newlength{\templenb}
\newsavebox{\tempboxa}
\newcommand{\shadebox}[1]{{\setlength{\fboxsep}{\graymargin pt}%
     \savebox{\tempboxa}{#1}%
     \settoheight{\templena}{\usebox{\tempboxa}}%
     \settodepth{\templenb}{\usebox{\tempboxa}}%
     \hspace*{-\fboxsep}\raisebox{0pt}[\templena][\templenb]%
        {\colorbox{boxshade}{\usebox{\tempboxa}}}\hspace*{-\fboxsep}}}

% \vshade{n} makes an n-point inter-paragraph space, with
%  shading if the `shading' flag is true.
\newlength{\vshadelen}
\setlength{\vshadelen}{0pt}
\newcommand{\vshade}[1]{\ifthenelse{\boolean{shading}}%
   {\global\setlength{\vshadelen}{#1pt}}%
   {\vspace{#1pt}}}

\newlength{\boxwidth}
\newlength{\multicommentdepth}

%%%%%%%%%%%%%%%%%%%%%%%%%%%%%%%%%%%%%%%%%%%%%%%%%%%%%%%%%%%%%%%%%%%%%%%%%%%%%
% THE cpar ENVIRONMENT                                                      %
% ^^^^^^^^^^^^^^^^^^^^                                                      %
% The LaTeX input                                                           %
%                                                                           %
%   \begin{cpar}{pop}{nest}{isLabel}{d}{e}{arg6}                            %
%     XXXXXXXXXXXXXXX                                                       %
%     XXXXXXXXXXXXXXX                                                       %
%     XXXXXXXXXXXXXXX                                                       %
%   \end{cpar}                                                              %
%                                                                           %
% produces one of two possible results.  If isLabel is the letter "T",      %
% it produces the following, where [label] is the result of typesetting     %
% arg6 in an LR box, and d is is a number representing a distance in        %
% points.                                                                   %
%                                                                           %
%   prevailing |<-- d -->[label]<- e ->XXXXXXXXXXXXXXX                      %
%         left |                       XXXXXXXXXXXXXXX                      %
%       margin |                       XXXXXXXXXXXXXXX                      %
%                                                                           %
% If isLabel is the letter "F", then it produces                            %
%                                                                           %
%   prevailing |<-- d -->XXXXXXXXXXXXXXXXXXXXXXX                            %
%         left |         <- e ->XXXXXXXXXXXXXXXX                            %
%       margin |                XXXXXXXXXXXXXXXX                            %
%                                                                           %
% where d and e are numbers representing distances in points.               %
%                                                                           %
% The prevailing left margin is the one in effect before the most recent    %
% pop (argument 1) cpar environments with "T" as the nest argument, where   %
% pop is a number \geq 0.                                                   %
%                                                                           %
% If the nest argument is the letter "T", then the prevailing left          %
% margin is moved to the left of the second (and following) lines of        %
% X's.  Otherwise, the prevailing left margin is left unchanged.            %
%                                                                           %
% An \unnest{n} command moves the prevailing left margin to where it was    %
% before the most recent n cpar environments with "T" as the nesting        %
% argument.                                                                 %
%                                                                           %
% The environment leaves no vertical space above or below it, or between    %
% its paragraphs.  (TLATeX inserts the proper amount of vertical space.)    %
%%%%%%%%%%%%%%%%%%%%%%%%%%%%%%%%%%%%%%%%%%%%%%%%%%%%%%%%%%%%%%%%%%%%%%%%%%%%%

\newcounter{pardepth}
\setcounter{pardepth}{0}

% \setgmargin{txt} defines \gmarginN to be txt, where N is \roman{pardepth}.
% \thegmargin equals \gmarginN, where N is \roman{pardepth}.
\newcommand{\setgmargin}[1]{%
  \expandafter\xdef\csname gmargin\roman{pardepth}\endcsname{#1}}
\newcommand{\thegmargin}{\csname gmargin\roman{pardepth}\endcsname}
\newcommand{\gmargin}{0pt}

\newsavebox{\tempsbox}

\newlength{\@cparht}
\newlength{\@cpardp}
\newenvironment{cpar}[6]{%
  \addtocounter{pardepth}{-#1}%
  \ifthenelse{\boolean{shading}}{\par\begin{lrbox}{\tempsbox}%
                                 \begin{minipage}[t]{\linewidth}}{}%
  \begin{list}{}{%
     \edef\temp{\thegmargin}
     \ifthenelse{\equal{#3}{T}}%
       {\settowidth{\leftmargin}{\hspace{\temp}\footnotesize #6\hspace{#5pt}}%
        \addtolength{\leftmargin}{#4pt}}%
       {\setlength{\leftmargin}{#4pt}%
        \addtolength{\leftmargin}{#5pt}%
        \addtolength{\leftmargin}{\temp}%
        \setlength{\itemindent}{-#5pt}}%
      \ifthenelse{\equal{#2}{T}}{\addtocounter{pardepth}{1}%
                                 \setgmargin{\the\leftmargin}}{}%
      \setlength{\labelwidth}{0pt}%
      \setlength{\labelsep}{0pt}%
      \setlength{\itemindent}{-\leftmargin}%
      \setlength{\topsep}{0pt}%
      \setlength{\parsep}{0pt}%
      \setlength{\partopsep}{0pt}%
      \setlength{\parskip}{0pt}%
      \setlength{\itemsep}{0pt}
      \setlength{\itemindent}{#4pt}%
      \addtolength{\itemindent}{-\leftmargin}}%
   \ifthenelse{\equal{#3}{T}}%
      {\item[\tstrut\footnotesize \hspace{\temp}{#6}\hspace{#5pt}]
        }%
      {\item[\tstrut\hspace{\temp}]%
         }%
   \footnotesize}
 {\hspace{-\the\lastskip}\tstrut
 \end{list}%
  \ifthenelse{\boolean{shading}}%
          {\end{minipage}%
           \end{lrbox}%
           \ifpcalshading
             \setlength{\@cparht}{\ht\tempsbox}%
             \setlength{\@cpardp}{\dp\tempsbox}%
             \addtolength{\@cparht}{.15em}%
             \addtolength{\@cpardp}{.2em}%
             \addtolength{\@cparht}{\@cpardp}%
            % I don't know what's going on here.  I want to add a
            % \pcalvspace high shaded line, but I don't know how to
            % do it.  A little trial and error shows that the following
            % does a reasonable job approximating that, eliminating
            % the line if \pcalvspace is small.
            \addtolength{\@cparht}{\pcalvspace}%
             \ifdim \pcalvspace > .8em
               \addtolength{\pcalvspace}{-.2em}%
               \hspace*{-\lcomindent}%
               \shadebox{\rule{0pt}{\pcalvspace}\hspace*{\textwidth}}\par
               \global\setlength{\pcalvspace}{0pt}%
               \fi
             \hspace*{-\lcomindent}%
             \makebox[0pt][l]{\raisebox{-\@cpardp}[0pt][0pt]{%
                 \shadebox{\rule{0pt}{\@cparht}\hspace*{\textwidth}}}}%
             \hspace*{\lcomindent}\usebox{\tempsbox}%
             \par
           \else
             \shadebox{\usebox{\tempsbox}}\par
           \fi}%
           {}%
  }

%%%%%%%%%%%%%%%%%%%%%%%%%%%%%%%%%%%%%%%%%%%%%%%%%%%%%%%%%%%%%%%%%%%%%%%%%%%%%%
% THE ppar ENVIRONMENT                                                       %
% ^^^^^^^^^^^^^^^^^^^^                                                       %
% The environment                                                            %
%                                                                            %
%   \begin{ppar} ... \end{ppar}                                              %
%                                                                            %
% is equivalent to                                                           %
%                                                                            %
%   \begin{cpar}{0}{F}{F}{0}{0}{} ... \end{cpar}                             %
%                                                                            %
% The environment is put around each line of the output for a PlusCal        %
% algorithm.                                                                 %
%%%%%%%%%%%%%%%%%%%%%%%%%%%%%%%%%%%%%%%%%%%%%%%%%%%%%%%%%%%%%%%%%%%%%%%%%%%%%%
%\newenvironment{ppar}{%
%  \ifthenelse{\boolean{shading}}{\par\begin{lrbox}{\tempsbox}%
%                                 \begin{minipage}[t]{\linewidth}}{}%
%  \begin{list}{}{%
%     \edef\temp{\thegmargin}
%        \setlength{\leftmargin}{0pt}%
%        \addtolength{\leftmargin}{\temp}%
%        \setlength{\itemindent}{0pt}%
%      \setlength{\labelwidth}{0pt}%
%      \setlength{\labelsep}{0pt}%
%      \setlength{\itemindent}{-\leftmargin}%
%      \setlength{\topsep}{0pt}%
%      \setlength{\parsep}{0pt}%
%      \setlength{\partopsep}{0pt}%
%      \setlength{\parskip}{0pt}%
%      \setlength{\itemsep}{0pt}
%      \setlength{\itemindent}{0pt}%
%      \addtolength{\itemindent}{-\leftmargin}}%
%      \item[\tstrut\hspace{\temp}]}%
% {\hspace{-\the\lastskip}\tstrut
% \end{list}%
%  \ifthenelse{\boolean{shading}}{\end{minipage}  
%                                 \end{lrbox}%
%                                 \shadebox{\usebox{\tempsbox}}\par}{}%
%  }

 %%% TESTING
 \newcommand{\xtest}[1]{\par
 \makebox[0pt][l]{\shadebox{\xtstrut\hspace*{\textwidth}}}%
 \mbox{$\mbox{}#1\mbox{}$}} 

% \newcommand{\xxtest}[1]{\par
% \makebox[0pt][l]{\shadebox{\xtstrut{#1}\hspace*{\textwidth}}}%
% \mbox{$\mbox{}#1\mbox{}$}} 

%\newlength{\pcalvspace}
%\setlength{\pcalvspace}{0pt}
% \newlength{\xxtestlen}
% \setlength{\xxtestlen}{0pt}
% \newcommand\xtstrut%
%   {\setlength{\xxtestlen}{1.15em}%
%    \addtolength{\xxtestlen}{\pcalvspace}%
%     \raisebox{\vshadelen}{\raisebox{-.25em}{\rule{0pt}{\xxtestlen}}}%
%    \global\setlength{\vshadelen}{0pt}%
%    \global\setlength{\pcalvspace}{0pt}}
   
   %%%% TESTING
   
   %% The xcpar environment
   %%  Note: overloaded use of \pcalvspace for testing.
   %%
%   \newlength{\xcparht}%
%   \newlength{\xcpardp}%
   
%   \newenvironment{xcpar}[6]{%
%  \addtocounter{pardepth}{-#1}%
%  \ifthenelse{\boolean{shading}}{\par\begin{lrbox}{\tempsbox}%
%                                 \begin{minipage}[t]{\linewidth}}{}%
%  \begin{list}{}{%
%     \edef\temp{\thegmargin}%
%     \ifthenelse{\equal{#3}{T}}%
%       {\settowidth{\leftmargin}{\hspace{\temp}\footnotesize #6\hspace{#5pt}}%
%        \addtolength{\leftmargin}{#4pt}}%
%       {\setlength{\leftmargin}{#4pt}%
%        \addtolength{\leftmargin}{#5pt}%
%        \addtolength{\leftmargin}{\temp}%
%        \setlength{\itemindent}{-#5pt}}%
%      \ifthenelse{\equal{#2}{T}}{\addtocounter{pardepth}{1}%
%                                 \setgmargin{\the\leftmargin}}{}%
%      \setlength{\labelwidth}{0pt}%
%      \setlength{\labelsep}{0pt}%
%      \setlength{\itemindent}{-\leftmargin}%
%      \setlength{\topsep}{0pt}%
%      \setlength{\parsep}{0pt}%
%      \setlength{\partopsep}{0pt}%
%      \setlength{\parskip}{0pt}%
%      \setlength{\itemsep}{0pt}%
%      \setlength{\itemindent}{#4pt}%
%      \addtolength{\itemindent}{-\leftmargin}}%
%   \ifthenelse{\equal{#3}{T}}%
%      {\item[\xtstrut\footnotesize \hspace{\temp}{#6}\hspace{#5pt}]%
%        }%
%      {\item[\xtstrut\hspace{\temp}]%
%         }%
%   \footnotesize}
% {\hspace{-\the\lastskip}\tstrut
% \end{list}%
%  \ifthenelse{\boolean{shading}}{\end{minipage}  
%                                 \end{lrbox}%
%                                 \setlength{\xcparht}{\ht\tempsbox}%
%                                 \setlength{\xcpardp}{\dp\tempsbox}%
%                                 \addtolength{\xcparht}{.15em}%
%                                 \addtolength{\xcpardp}{.2em}%
%                                 \addtolength{\xcparht}{\xcpardp}%
%                                 \hspace*{-\lcomindent}%
%                                 \makebox[0pt][l]{\raisebox{-\xcpardp}[0pt][0pt]{%
%                                      \shadebox{\rule{0pt}{\xcparht}\hspace*{\textwidth}}}}%
%                                 \hspace*{\lcomindent}\usebox{\tempsbox}%
%                                 \par}{}%
%  }
%  
% \newlength{\xmcomlen}
%\newenvironment{xmcom}[1]{%
%  \setcounter{pardepth}{0}%
%  \hspace{.65em}%
%  \begin{lrbox}{\alignbox}\sloppypar%
%      \setboolean{shading}{false}%
%      \setlength{\boxwidth}{#1pt}%
%      \addtolength{\boxwidth}{-.65em}%
%      \begin{minipage}[t]{\boxwidth}\footnotesize
%      \parskip=0pt\relax}%
%       {\end{minipage}\end{lrbox}%
%       \setlength{\xmcomlen}{\textwidth}%
%       \addtolength{\xmcomlen}{-\wd\alignbox}%
%       \settodepth{\alignwidth}{\usebox{\alignbox}}%
%       \global\setlength{\multicommentdepth}{\alignwidth}%
%       \setlength{\boxwidth}{\alignwidth}%
%       \global\addtolength{\alignwidth}{-\maxdepth}%
%       \addtolength{\boxwidth}{.1em}%
%       \raisebox{0pt}[0pt][0pt]{%
%        \ifthenelse{\boolean{shading}}%
%          {\hspace*{-\xmcomlen}\shadebox{\rule[-\boxwidth]{0pt}{0pt}%
%                                 \hspace*{\xmcomlen}\usebox{\alignbox}}}%
%          {\usebox{\alignbox}}}%
%       \vspace*{\alignwidth}\pagebreak[0]\vspace{-\alignwidth}\par}
% % a multi-line comment, whose first argument is its width in points.
%  
   
%%%%%%%%%%%%%%%%%%%%%%%%%%%%%%%%%%%%%%%%%%%%%%%%%%%%%%%%%%%%%%%%%%%%%%%%%%%%%%
% THE lcom ENVIRONMENT                                                       %
% ^^^^^^^^^^^^^^^^^^^^                                                       %
% A multi-line comment with no text to its left is typeset in an lcom        % 
% environment, whose argument is a number representing the indentation       % 
% of the left margin, in points.  All the text of the comment should be      % 
% inside cpar environments.                                                  % 
%%%%%%%%%%%%%%%%%%%%%%%%%%%%%%%%%%%%%%%%%%%%%%%%%%%%%%%%%%%%%%%%%%%%%%%%%%%%%%
\newenvironment{lcom}[1]{%
  \setlength{\lcomindent}{#1pt} % Added for PlusCal handling.
  \par\vspace{.2em}%
  \sloppypar
  \setcounter{pardepth}{0}%
  \footnotesize
  \begin{list}{}{%
    \setlength{\leftmargin}{#1pt}
    \setlength{\labelwidth}{0pt}%
    \setlength{\labelsep}{0pt}%
    \setlength{\itemindent}{0pt}%
    \setlength{\topsep}{0pt}%
    \setlength{\parsep}{0pt}%
    \setlength{\partopsep}{0pt}%
    \setlength{\parskip}{0pt}}
    \item[]}%
  {\end{list}\vspace{.3em}\setlength{\lcomindent}{0pt}%
 }


%%%%%%%%%%%%%%%%%%%%%%%%%%%%%%%%%%%%%%%%%%%%%%%%%%%%%%%%%%%%%%%%%%%%%%%%%%%%%
% THE mcom ENVIRONMENT AND \mutivspace COMMAND                              %
% ^^^^^^^^^^^^^^^^^^^^^^^^^^^^^^^^^^^^^^^^^^^^                              %
%                                                                           %
% A part of the spec containing a right-comment of the form                 %
%                                                                           %
%      xxxx (*************)                                                 %
%      yyyy (* ccccccccc *)                                                 %
%      ...  (* ccccccccc *)                                                 %
%           (* ccccccccc *)                                                 %
%           (* ccccccccc *)                                                 %
%           (*************)                                                 %
%                                                                           %
% is typeset by                                                             %
%                                                                           %
%     XXXX \begin{mcom}{d}                                                  %
%            CCCC ... CCC                                                   %
%          \end{mcom}                                                       %
%     YYYY ...                                                              %
%     \multivspace{n}                                                       %
%                                                                           %
% where the number d is the width in points of the comment, n is the        %
% number of xxxx, yyyy, ...  lines to the left of the comment.              %
% All the text of the comment should be typeset in cpar environments.       %
%                                                                           %
% This puts the comment into a single box (so no page breaks can occur      %
% within it).  The entire box is shaded iff the shading flag is true.       %
%%%%%%%%%%%%%%%%%%%%%%%%%%%%%%%%%%%%%%%%%%%%%%%%%%%%%%%%%%%%%%%%%%%%%%%%%%%%%
\newlength{\xmcomlen}%
\newenvironment{mcom}[1]{%
  \setcounter{pardepth}{0}%
  \hspace{.65em}%
  \begin{lrbox}{\alignbox}\sloppypar%
      \setboolean{shading}{false}%
      \setlength{\boxwidth}{#1pt}%
      \addtolength{\boxwidth}{-.65em}%
      \begin{minipage}[t]{\boxwidth}\footnotesize
      \parskip=0pt\relax}%
       {\end{minipage}\end{lrbox}%
       \setlength{\xmcomlen}{\textwidth}%       % For PlusCal shading
       \addtolength{\xmcomlen}{-\wd\alignbox}%  % For PlusCal shading
       \settodepth{\alignwidth}{\usebox{\alignbox}}%
       \global\setlength{\multicommentdepth}{\alignwidth}%
       \setlength{\boxwidth}{\alignwidth}%      % For PlusCal shading
       \global\addtolength{\alignwidth}{-\maxdepth}%
       \addtolength{\boxwidth}{.1em}%           % For PlusCal shading
      \raisebox{0pt}[0pt][0pt]{%
        \ifthenelse{\boolean{shading}}%
          {\ifpcalshading
             \hspace*{-\xmcomlen}%
             \shadebox{\rule[-\boxwidth]{0pt}{0pt}\hspace*{\xmcomlen}%
                          \usebox{\alignbox}}%
           \else
             \shadebox{\usebox{\alignbox}}
           \fi
          }%
          {\usebox{\alignbox}}}%
       \vspace*{\alignwidth}\pagebreak[0]\vspace{-\alignwidth}\par}
 % a multi-line comment, whose first argument is its width in points.


% \multispace{n} produces the vertical space indicated by "|"s in 
% this situation
%   
%     xxxx (*************)
%     xxxx (* ccccccccc *)
%      |   (* ccccccccc *)
%      |   (* ccccccccc *)
%      |   (* ccccccccc *)
%      |   (*************)
%
% where n is the number of "xxxx" lines.
\newcommand{\multivspace}[1]{\addtolength{\multicommentdepth}{-#1\baselineskip}%
 \addtolength{\multicommentdepth}{1.2em}%
 \ifthenelse{\lengthtest{\multicommentdepth > 0pt}}%
    {\par\vspace{\multicommentdepth}\par}{}}

%\newenvironment{hpar}[2]{%
%  \begin{list}{}{\setlength{\leftmargin}{#1pt}%
%                 \addtolength{\leftmargin}{#2pt}%
%                 \setlength{\itemindent}{-#2pt}%
%                 \setlength{\topsep}{0pt}%
%                 \setlength{\parsep}{0pt}%
%                 \setlength{\partopsep}{0pt}%
%                 \setlength{\parskip}{0pt}%
%                 \addtolength{\labelsep}{0pt}}%
%  \item[]\footnotesize}{\end{list}}
%    %%%%%%%%%%%%%%%%%%%%%%%%%%%%%%%%%%%%%%%%%%%%%%%%%%%%%%%%%%%%%%%%%%%%%%%%
%    % Typesets a sequence of paragraphs like this:                         %
%    %                                                                      %
%    %      left |<-- d1 --> XXXXXXXXXXXXXXXXXXXXXXXX                       %
%    %    margin |           <- d2 -> XXXXXXXXXXXXXXX                       %
%    %           |                    XXXXXXXXXXXXXXX                       %
%    %           |                                                          %
%    %           |                    XXXXXXXXXXXXXXX                       %
%    %           |                    XXXXXXXXXXXXXXX                       %
%    %                                                                      %
%    % where d1 = #1pt and d2 = #2pt, but with no vspace between            %
%    % paragraphs.                                                          %
%    %%%%%%%%%%%%%%%%%%%%%%%%%%%%%%%%%%%%%%%%%%%%%%%%%%%%%%%%%%%%%%%%%%%%%%%%

%%%%%%%%%%%%%%%%%%%%%%%%%%%%%%%%%%%%%%%%%%%%%%%%%%%%%%%%%%%%%%%%%%%%%%
% Commands for repeated characters that produce dashes.              %
%%%%%%%%%%%%%%%%%%%%%%%%%%%%%%%%%%%%%%%%%%%%%%%%%%%%%%%%%%%%%%%%%%%%%%
% \raisedDash{wd}{ht}{thk} makes a horizontal line wd characters wide, 
% raised a distance ht ex's above the baseline, with a thickness of 
% thk em's.
\newcommand{\raisedDash}[3]{\raisebox{#2ex}{\setlength{\alignwidth}{.5em}%
  \rule{#1\alignwidth}{#3em}}}

% The following commands take a single argument n and produce the
% output for n repeated characters, as follows
%   \cdash:    -
%   \tdash:    ~
%   \ceqdash:  =
%   \usdash:   _
\newcommand{\cdash}[1]{\raisedDash{#1}{.5}{.04}}
\newcommand{\usdash}[1]{\raisedDash{#1}{0}{.04}}
\newcommand{\ceqdash}[1]{\raisedDash{#1}{.5}{.08}}
\newcommand{\tdash}[1]{\raisedDash{#1}{1}{.08}}

\newlength{\spacewidth}
\setlength{\spacewidth}{.2em}
\newcommand{\e}[1]{\hspace{#1\spacewidth}}
%% \e{i} produces space corresponding to i input spaces.


%% Alignment-file Commands

\newlength{\alignboxwidth}
\newlength{\alignwidth}
\newsavebox{\alignbox}

% \al{i}{j}{txt} is used in the alignment file to put "%{i}{j}{wd}"
% in the log file, where wd is the width of the line up to that point,
% and txt is the following text.
\newcommand{\al}[3]{%
  \typeout{\%{#1}{#2}{\the\alignwidth}}%
  \cl{#3}}

%% \cl{txt} continues a specification line in the alignment file
%% with text txt.
\newcommand{\cl}[1]{%
  \savebox{\alignbox}{\mbox{$\mbox{}#1\mbox{}$}}%
  \settowidth{\alignboxwidth}{\usebox{\alignbox}}%
  \addtolength{\alignwidth}{\alignboxwidth}%
  \usebox{\alignbox}}

% \fl{txt} in the alignment file begins a specification line that
% starts with the text txt.
\newcommand{\fl}[1]{%
  \par
  \savebox{\alignbox}{\mbox{$\mbox{}#1\mbox{}$}}%
  \settowidth{\alignwidth}{\usebox{\alignbox}}%
  \usebox{\alignbox}}



  
%%%%%%%%%%%%%%%%%%%%%%%%%%%%%%%%%%%%%%%%%%%%%%%%%%%%%%%%%%%%%%%%%%%%%%%%%%%%%
% Ordinarily, TeX typesets letters in math mode in a special math italic    %
% font.  This makes it typeset "it" to look like the product of the         %
% variables i and t, rather than like the word "it".  The following         %
% commands tell TeX to use an ordinary italic font instead.                 %
%%%%%%%%%%%%%%%%%%%%%%%%%%%%%%%%%%%%%%%%%%%%%%%%%%%%%%%%%%%%%%%%%%%%%%%%%%%%%
\ifx\documentclass\undefined
\else
  \DeclareSymbolFont{tlaitalics}{\encodingdefault}{cmr}{m}{it}
  \let\itfam\symtlaitalics
\fi

\makeatletter
\newcommand{\tlx@c}{\c@tlx@ctr\advance\c@tlx@ctr\@ne}
\newcounter{tlx@ctr}
\c@tlx@ctr=\itfam \multiply\c@tlx@ctr"100\relax \advance\c@tlx@ctr "7061\relax
\mathcode`a=\tlx@c \mathcode`b=\tlx@c \mathcode`c=\tlx@c \mathcode`d=\tlx@c
\mathcode`e=\tlx@c \mathcode`f=\tlx@c \mathcode`g=\tlx@c \mathcode`h=\tlx@c
\mathcode`i=\tlx@c \mathcode`j=\tlx@c \mathcode`k=\tlx@c \mathcode`l=\tlx@c
\mathcode`m=\tlx@c \mathcode`n=\tlx@c \mathcode`o=\tlx@c \mathcode`p=\tlx@c
\mathcode`q=\tlx@c \mathcode`r=\tlx@c \mathcode`s=\tlx@c \mathcode`t=\tlx@c
\mathcode`u=\tlx@c \mathcode`v=\tlx@c \mathcode`w=\tlx@c \mathcode`x=\tlx@c
\mathcode`y=\tlx@c \mathcode`z=\tlx@c
\c@tlx@ctr=\itfam \multiply\c@tlx@ctr"100\relax \advance\c@tlx@ctr "7041\relax
\mathcode`A=\tlx@c \mathcode`B=\tlx@c \mathcode`C=\tlx@c \mathcode`D=\tlx@c
\mathcode`E=\tlx@c \mathcode`F=\tlx@c \mathcode`G=\tlx@c \mathcode`H=\tlx@c
\mathcode`I=\tlx@c \mathcode`J=\tlx@c \mathcode`K=\tlx@c \mathcode`L=\tlx@c
\mathcode`M=\tlx@c \mathcode`N=\tlx@c \mathcode`O=\tlx@c \mathcode`P=\tlx@c
\mathcode`Q=\tlx@c \mathcode`R=\tlx@c \mathcode`S=\tlx@c \mathcode`T=\tlx@c
\mathcode`U=\tlx@c \mathcode`V=\tlx@c \mathcode`W=\tlx@c \mathcode`X=\tlx@c
\mathcode`Y=\tlx@c \mathcode`Z=\tlx@c
\makeatother

%%%%%%%%%%%%%%%%%%%%%%%%%%%%%%%%%%%%%%%%%%%%%%%%%%%%%%%%%%
%                THE describe ENVIRONMENT                %
%%%%%%%%%%%%%%%%%%%%%%%%%%%%%%%%%%%%%%%%%%%%%%%%%%%%%%%%%%
%
%
% It is like the description environment except it takes an argument
% ARG that should be the text of the widest label.  It adjusts the
% indentation so each item with label LABEL produces
%%      LABEL             blah blah blah
%%      <- width of ARG ->blah blah blah
%%                        blah blah blah
\newenvironment{describe}[1]%
   {\begin{list}{}{\settowidth{\labelwidth}{#1}%
            \setlength{\labelsep}{.5em}%
            \setlength{\leftmargin}{\labelwidth}% 
            \addtolength{\leftmargin}{\labelsep}%
            \addtolength{\leftmargin}{\parindent}%
            \def\makelabel##1{\rm ##1\hfill}}%
            \setlength{\topsep}{0pt}}%% 
                % Sets \topsep to 0 to reduce vertical space above
                % and below embedded displayed equations
   {\end{list}}

%   For tlatex.TeX
\usepackage{verbatim}
\makeatletter
\def\tla{\let\%\relax%
         \@bsphack
         \typeout{\%{\the\linewidth}}%
             \let\do\@makeother\dospecials\catcode`\^^M\active
             \let\verbatim@startline\relax
             \let\verbatim@addtoline\@gobble
             \let\verbatim@processline\relax
             \let\verbatim@finish\relax
             \verbatim@}
\let\endtla=\@esphack

\let\pcal=\tla
\let\endpcal=\endtla
\let\ppcal=\tla
\let\endppcal=\endtla

% The tlatex environment is used by TLATeX.TeX to typeset TLA+.
% TLATeX.TLA starts its files by writing a \tlatex command.  This
% command/environment sets \parindent to 0 and defines \% to its
% standard definition because the writing of the log files is messed up
% if \% is defined to be something else.  It also executes
% \@computerule to determine the dimensions for the TLA horizonatl
% bars.
\newenvironment{tlatex}{\@computerule%
                        \setlength{\parindent}{0pt}%
                       \makeatletter\chardef\%=`\%}{}


% The notla environment produces no output.  You can turn a 
% tla environment to a notla environment to prevent tlatex.TeX from
% re-formatting the environment.

\def\notla{\let\%\relax%
         \@bsphack
             \let\do\@makeother\dospecials\catcode`\^^M\active
             \let\verbatim@startline\relax
             \let\verbatim@addtoline\@gobble
             \let\verbatim@processline\relax
             \let\verbatim@finish\relax
             \verbatim@}
\let\endnotla=\@esphack

\let\nopcal=\notla
\let\endnopcal=\endnotla
\let\noppcal=\notla
\let\endnoppcal=\endnotla

%%%%%%%%%%%%%%%%%%%%%%%% end of tlatex.sty file %%%%%%%%%%%%%%%%%%%%%%% 
% last modified on Fri  3 August 2012 at 14:23:49 PST by lamport

\begin{document}
\tlatex
\setboolean{shading}{true}
\fl{}\moduleLeftDash\cl{ {\MODULE} raft}\moduleRightDash\cl{}
\fl{}
\fl{}
\fl{}
\fl{}
\fl{}
\fl{}
\fl{ {\EXTENDS} Naturals ,\, FiniteSets ,\, Sequences ,\, TLC}
\fl{}
\fl{}
\fl{ {\CONSTANTS} Server}
\fl{}
\fl{}
\fl{ {\CONSTANTS} Value}
\fl{}
\fl{}
\fl{ {\CONSTANTS} Follower ,\, Candidate ,\, Leader}
\fl{}
\fl{}
\fl{ {\CONSTANTS} Nil}
\fl{}
\fl{}
\fl{ {\CONSTANTS}}\al{22}{1}{ RequestVoteRequest ,\, RequestVoteResponse ,\,}
\fl{ AppendEntriesRequest ,\, AppendEntriesResponse}
\fl{}
\fl{}\midbar\cl{}
\fl{}
\fl{}
\fl{}
\fl{}
\fl{}
\fl{ {\VARIABLE} messages}
\fl{}
\fl{}
\fl{}
\fl{}
\fl{}
\fl{}
\fl{ {\VARIABLE} elections}
\fl{}
\fl{}
\fl{}
\fl{}
\fl{ {\VARIABLE} allLogs}
\fl{}
\fl{}
\fl{}
\fl{}
\fl{}
\fl{}
\fl{}
\fl{}
\fl{}
\fl{}
\fl{}\midbar\cl{}
\fl{}
\fl{}
\fl{}
\fl{ {\VARIABLE} currentTerm}
\fl{}
\fl{ {\VARIABLE} state}
\fl{}
\fl{}
\fl{ {\VARIABLE} votedFor}
 \fl{ serverVars \.{\defeq} {\langle} currentTerm ,\, state ,\, votedFor
 {\rangle}}
\fl{}
\fl{}
\fl{}
\fl{}
\fl{ {\VARIABLE} log}
\fl{}
\fl{ {\VARIABLE} commitIndex}
\fl{ logVars \.{\defeq} {\langle} log ,\, commitIndex {\rangle}}
\fl{}
\fl{}
\fl{}
\fl{}
\fl{ {\VARIABLE} votesResponded}
\fl{}
\fl{}
\fl{ {\VARIABLE} votesGranted}
\fl{}
\fl{}
\fl{}
\fl{}
\fl{ {\VARIABLE} voterLog}
 \fl{ candidateVars \.{\defeq} {\langle} votesResponded ,\, votesGranted ,\,
 voterLog {\rangle}}
\fl{}
\fl{}
\fl{}
\fl{ {\VARIABLE} nextIndex}
\fl{}
\fl{}
\fl{ {\VARIABLE} matchIndex}
 \fl{ leaderVars \.{\defeq} {\langle} nextIndex ,\, matchIndex ,\, elections
 {\rangle}}
\fl{}
\fl{}
\fl{}\midbar\cl{}
\fl{}
\fl{}
 \fl{ vars \.{\defeq} {\langle} messages ,\, allLogs ,\, serverVars ,\,
 candidateVars ,\, leaderVars ,\, logVars {\rangle}}
\fl{}
\fl{}
\fl{}
\fl{}
\fl{}
\fl{}
\fl{}
\fl{}
\fl{}
\fl{}
\fl{}\midbar\cl{}
\fl{}
\fl{}
\fl{}
\fl{}
\fl{}
\fl{}
 \fl{ Quorum \.{\defeq} \{ i \.{\in} {\SUBSET} ( Server ) \.{:} Cardinality (
 i ) \.{*} 2 \.{>} Cardinality ( Server ) \}}
\fl{}
\fl{}
 \fl{ LastTerm ( xlog ) \.{\defeq} {\IF} Len ( xlog ) \.{=} 0 \.{\THEN} 0
 \.{\ELSE} xlog [ Len ( xlog ) ] . term}
\fl{}
\fl{}
\fl{}
\fl{}
\fl{}
\fl{ WithMessage ( m ,\, msgs ) \.{\defeq}}
\fl{ {\IF}}\al{128}{1}{ m \.{\in} {\DOMAIN} msgs \.{\THEN}}
\fl{ [ msgs {\EXCEPT} {\bang} [ m ] \.{=} msgs [ m ] \.{+} 1 ]}
\fl{ \.{\ELSE}}
\fl{ msgs \.{\,@@\,} ( m \.{\colongt} 1 )}
\fl{}
\fl{}
\fl{}
\fl{}
\fl{}
\fl{ WithoutMessage ( m ,\, msgs ) \.{\defeq}}
\fl{ {\IF}}\al{138}{1}{ m \.{\in} {\DOMAIN} msgs \.{\THEN}}
\fl{ [ msgs {\EXCEPT} {\bang} [ m ] \.{=} msgs [ m ] \.{-} 1 ]}
\fl{ \.{\ELSE}}
\fl{ msgs}
\fl{}
\fl{}
\fl{}
\fl{}
 \fl{ Send ( m ) \.{\defeq} messages \.{'} \.{=} WithMessage ( m ,\, messages
 )}
\fl{}
\fl{}
\fl{}
\fl{}
\fl{}
 \fl{ Discard ( m ) \.{\defeq} messages \.{'} \.{=} WithoutMessage ( m ,\,
 messages )}
\fl{}
\fl{}
\fl{}
\fl{}
\fl{ Reply ( response ,\, request ) \.{\defeq}}
 \fl{ messages \.{'} \.{=} WithoutMessage ( request ,\, WithMessage ( response
 ,\, messages ) )}
\fl{}
\fl{}
\fl{}
 \fl{ Min ( s ) \.{\defeq} {\CHOOSE} x \.{\in} s \.{:} \A\, y \.{\in} s \.{:}
 x \.{\leq} y}
\fl{}
 \fl{ Max ( s ) \.{\defeq} {\CHOOSE} x \.{\in} s \.{:} \A\, y \.{\in} s \.{:}
 x \.{\geq} y}
\fl{}
\fl{}
\fl{}
\fl{}
\fl{}
\fl{}
\fl{}
\fl{}
\fl{}
\fl{}
\fl{}
\fl{}
\fl{}
\fl{}
\fl{}
\fl{}
\fl{}
\fl{}
\fl{}
\fl{}\midbar\cl{}
\fl{}
\fl{}
\fl{}
 \fl{ InitHistoryVars \.{\defeq}}\al{188}{2}{ \.{\land}}\al{188}{3}{
 elections}\al{188}{4}{ \.{=}}\al{188}{5}{ \{ \}}
\fl{ \.{\land}}\al{189}{1}{ allLogs}\al{189}{2}{ \.{=}}\al{189}{3}{ \{ \}}
 \fl{ \.{\land}}\al{190}{1}{ voterLog}\al{190}{2}{ \.{=} [ i \.{\in} Server
 \.{\mapsto} [ j \.{\in} \{ \} \.{\mapsto} {\langle} {\rangle} ] ]}
\fl{}
\fl{}
\fl{}
 \fl{ InitServerVars \.{\defeq}}\al{194}{2}{ \.{\land}}\al{194}{3}{
 currentTerm}\al{194}{4}{ \.{=}}\al{194}{5}{ [ i}\al{194}{7}{
 \.{\in}}\al{194}{8}{ Server}\al{194}{9}{ \.{\mapsto}}\al{194}{10}{ 1 ]}
 \fl{ \.{\land}}\al{195}{1}{ state}\al{195}{2}{ \.{=}}\al{195}{3}{ [
 i}\al{195}{5}{ \.{\in}}\al{195}{6}{ Server}\al{195}{7}{
 \.{\mapsto}}\al{195}{8}{ Follower ]}
 \fl{ \.{\land}}\al{196}{1}{ votedFor}\al{196}{2}{ \.{=} [ i}\al{196}{5}{
 \.{\in} Server}\al{196}{7}{ \.{\mapsto} Nil ]}
\fl{}
\fl{}
 \fl{ InitCandidateVars \.{\defeq}}\al{199}{2}{ \.{\land}}\al{199}{3}{
 votesResponded}\al{199}{4}{ \.{=}}\al{199}{5}{ [ i}\al{199}{7}{
 \.{\in}}\al{199}{8}{ Server}\al{199}{9}{ \.{\mapsto}}\al{199}{10}{ \{ \} ]}
 \fl{ \.{\land}}\al{200}{1}{ votesGranted}\al{200}{2}{ \.{=}}\al{200}{3}{ [
 i}\al{200}{5}{ \.{\in}}\al{200}{6}{ Server}\al{200}{7}{
 \.{\mapsto}}\al{200}{8}{ \{ \} ]}
\fl{}
\fl{}
\fl{}
\fl{}
\fl{}
\fl{}
 \fl{ InitLeaderVars \.{\defeq}}\al{207}{2}{ \.{\land}}\al{207}{3}{
 nextIndex}\al{207}{4}{ \.{=}}\al{207}{5}{ [ i}\al{207}{7}{
 \.{\in}}\al{207}{8}{ Server}\al{207}{9}{ \.{\mapsto}}\al{207}{10}{ [
 j}\al{207}{12}{ \.{\in}}\al{207}{13}{ Server}\al{207}{14}{
 \.{\mapsto}}\al{207}{15}{ 1 ] ]}
 \fl{ \.{\land}}\al{208}{1}{ matchIndex}\al{208}{2}{ \.{=}}\al{208}{3}{ [
 i}\al{208}{5}{ \.{\in}}\al{208}{6}{ Server}\al{208}{7}{
 \.{\mapsto}}\al{208}{8}{ [ j}\al{208}{10}{ \.{\in}}\al{208}{11}{
 Server}\al{208}{12}{ \.{\mapsto}}\al{208}{13}{ 0 ] ]}
\fl{}
\fl{}
\fl{}
 \fl{ InitLogVars \.{\defeq}}\al{212}{2}{ \.{\land}}\al{212}{3}{
 log}\al{212}{4}{ \.{=}}\al{212}{5}{ [ i}\al{212}{7}{ \.{\in}}\al{212}{8}{
 Server}\al{212}{9}{ \.{\mapsto}}\al{212}{10}{ {\langle} {\rangle} ]}
 \fl{ \.{\land}}\al{213}{1}{ commitIndex}\al{213}{2}{ \.{=}}\al{213}{3}{ [
 i}\al{213}{5}{ \.{\in}}\al{213}{6}{ Server}\al{213}{7}{
 \.{\mapsto}}\al{213}{8}{ 0 ]}
\fl{}
\fl{}
\fl{}
 \fl{ Init \.{\defeq}}\al{217}{2}{ \.{\land}}\al{217}{3}{ messages \.{=} [ m
 \.{\in} \{ \} \.{\mapsto} 0 ]}
\fl{ \.{\land}}\al{218}{1}{ InitHistoryVars}
\fl{ \.{\land}}\al{219}{1}{ InitServerVars}
\fl{ \.{\land}}\al{220}{1}{ InitCandidateVars}
\fl{ \.{\land}}\al{221}{1}{ InitLeaderVars}
\fl{ \.{\land}}\al{222}{1}{ InitLogVars}
\fl{}
\fl{}
\fl{}
\fl{}
\fl{}
\fl{}\midbar\cl{}
\fl{}
\fl{}
\fl{}
\fl{}
\fl{}
\fl{}
\fl{ Restart}\al{235}{1}{ ( i ) \.{\defeq}}
 \fl{ \.{\land}}\al{236}{1}{ state \.{'}}\al{236}{3}{ \.{=}}\al{236}{4}{ [
 state {\EXCEPT} {\bang} [ i ] \.{=} Follower ]}\al{236}{14}{%
\@y{\@s{0}%
 状�?�变更为跟随�?
}%
}
 \fl{ \.{\land}}\al{237}{1}{ votesResponded \.{'}}\al{237}{3}{
 \.{=}}\al{237}{4}{ [ votesResponded {\EXCEPT} {\bang} [ i ] \.{=} \{ \}
 ]}\al{237}{15}{%
\@y{\@s{0}%
 清空该节点投票响应记�?
}%
}
 \fl{ \.{\land}}\al{238}{1}{ votesGranted \.{'}}\al{238}{3}{ \.{=} [
 votesGranted {\EXCEPT} {\bang} [ i ] \.{=} \{ \} ]}\al{238}{15}{%
\@y{\@s{0}%
 清空该节点的投票授予记录
}%
}
 \fl{ \.{\land}}\al{239}{1}{ voterLog \.{'}}\al{239}{3}{ \.{=} [ voterLog
 {\EXCEPT} {\bang} [ i ] \.{=} [ j \.{\in} \{ \} \.{\mapsto} {\langle}
 {\rangle} ] ]}\al{239}{22}{%
\@y{\@s{0}%
 清空投票日志
}%
}
 \fl{ \.{\land}}\al{240}{1}{ nextIndex \.{'}}\al{240}{3}{ \.{=} [ nextIndex
 {\EXCEPT} {\bang} [ i ] \.{=} [ j \.{\in} Server \.{\mapsto} 1 ]
 ]}\al{240}{20}{%
\@y{\@s{0}%
 重置\ensuremath{nextIndex
}}%
}
 \fl{ \.{\land}}\al{241}{1}{ matchIndex \.{'}}\al{241}{3}{ \.{=} [ matchIndex
 {\EXCEPT} {\bang} [ i ] \.{=} [ j \.{\in} Server \.{\mapsto} 0 ]
 ]}\al{241}{20}{%
\@y{\@s{0}%
 重置\ensuremath{matchIndex
}}%
}
 \fl{ \.{\land}}\al{242}{1}{ commitIndex \.{'}}\al{242}{3}{ \.{=} [
 commitIndex {\EXCEPT} {\bang} [ i ] \.{=} 0 ]}\al{242}{14}{%
\@y{\@s{0}%
 重置\ensuremath{commitIndex
}}%
}
 \fl{ \.{\land}}\al{243}{1}{ {\UNCHANGED} {\langle} messages ,\, currentTerm
 ,\, votedFor ,\, log ,\, elections {\rangle}}
\fl{}
\fl{}
\fl{}
\fl{}
 \fl{ Timeout ( i ) \.{\defeq}}\al{248}{5}{ \.{\land}}\al{248}{6}{ state [ i ]
 \.{\in} \{ Follower ,\, Candidate \}}
 \fl{ \.{\land}}\al{249}{1}{ state \.{'} \.{=} [ state {\EXCEPT} {\bang} [ i ]
 \.{=} Candidate ]}\al{249}{14}{%
\@y{\@s{0}%
 变更状�?�为候�?��??
}%
}
 \fl{ \.{\land}}\al{250}{1}{ currentTerm \.{'} \.{=} [ currentTerm {\EXCEPT}
 {\bang} [ i ] \.{=} currentTerm [ i ] \.{+} 1 ]}\al{250}{19}{%
\@y{\@s{0}%
 任期号自�?1
}%
}
\fl{}
\fl{}
 \fl{ \.{\land}}\al{253}{1}{ votedFor \.{'} \.{=} [ votedFor {\EXCEPT} {\bang}
 [ i ] \.{=} Nil ]}
 \fl{ \.{\land}}\al{254}{1}{ votesResponded \.{'}}\al{254}{3}{
 \.{=}}\al{254}{4}{ [ votesResponded {\EXCEPT} {\bang} [ i ] \.{=} \{ \}
 ]}\al{254}{15}{%
\@y{\@s{0}%
 清空该节点的投票响应记录
}%
}
 \fl{ \.{\land}}\al{255}{1}{ votesGranted \.{'}}\al{255}{3}{
 \.{=}}\al{255}{4}{ [ votesGranted {\EXCEPT} {\bang} [ i ] \.{=} \{ \} ]}
 \fl{ \.{\land}}\al{256}{1}{ voterLog \.{'}}\al{256}{3}{ \.{=} [ voterLog
 {\EXCEPT} {\bang} [ i ] \.{=} [ j \.{\in} \{ \} \.{\mapsto} {\langle}
 {\rangle} ] ]}\al{256}{22}{%
\@y{\@s{0}%
 重置投票日志
}%
}
 \fl{ \.{\land}}\al{257}{1}{ {\UNCHANGED} {\langle} messages ,\, leaderVars
 ,\, logVars {\rangle}}
\fl{}
\fl{}
\fl{}
\fl{}
\fl{ RequestVote ( i ,\, j ) \.{\defeq}}
\fl{ \.{\land}}\al{263}{1}{ state [ i ] \.{=} Candidate}
\fl{ \.{\land}}\al{264}{1}{ j \.{\notin} votesResponded [ i ]}\al{264}{7}{%
\@y{\@s{0}%
 节点\ensuremath{j}尚未给出过该轮投票的响应
}%
}
 \fl{ \.{\land}}\al{265}{1}{ Send ( [}\al{265}{4}{ mtype}\al{265}{5}{
 \.{\mapsto}}\al{265}{6}{ RequestVoteRequest ,\,}\al{265}{8}{%
\@y{\@s{0}%
 发�?�消�?
}%
}
\fl{ mterm}\al{266}{1}{ \.{\mapsto}}\al{266}{2}{ currentTerm [ i ] ,\,}
\fl{ mlastLogTerm}\al{267}{1}{ \.{\mapsto} LastTerm ( log [ i ] ) ,\,}
\fl{ mlastLogIndex}\al{268}{1}{ \.{\mapsto} Len ( log [ i ] ) ,\,}
\fl{ msource}\al{269}{1}{ \.{\mapsto} i ,\,}
\fl{ mdest}\al{270}{1}{ \.{\mapsto} j ] )}
 \fl{ \.{\land}}\al{271}{1}{ {\UNCHANGED} {\langle} serverVars ,\,
 candidateVars ,\, leaderVars ,\, logVars {\rangle}}
\fl{}
\fl{}
\fl{}
\fl{}
\fl{ AppendEntries ( i ,\, j ) \.{\defeq}}
\fl{ \.{\land}}\al{277}{1}{ i \.{\neq} j}
\fl{ \.{\land}}\al{278}{1}{ state [ i ] \.{=} Leader}
 \fl{ \.{\land}}\al{279}{1}{ \.{\LET}}\al{279}{2}{ prevLogIndex \.{\defeq}
 nextIndex [ i ] [ j ] \.{-} 1}
 \fl{ prevLogTerm \.{\defeq}}\al{280}{2}{ {\IF}}\al{280}{3}{ prevLogIndex
 \.{>} 0 \.{\THEN}}
\fl{ log [ i ] [ prevLogIndex ] . term}
\fl{ \.{\ELSE}}
\fl{ 0}
\fl{}
 \fl{ lastEntry \.{\defeq} Min ( \{ Len ( log [ i ] ) ,\, nextIndex [ i ] [ j
 ] \} )}
 \fl{ entries \.{\defeq} SubSeq ( log [ i ] ,\, nextIndex [ i ] [ j ] ,\,
 lastEntry )}
 \fl{ \.{\IN} Send ( [}\al{287}{4}{ mtype}\al{287}{5}{
 \.{\mapsto}}\al{287}{6}{ AppendEntriesRequest ,\,}
\fl{ mterm}\al{288}{1}{ \.{\mapsto}}\al{288}{2}{ currentTerm [ i ] ,\,}
\fl{ mprevLogIndex}\al{289}{1}{ \.{\mapsto} prevLogIndex ,\,}
\fl{ mprevLogTerm}\al{290}{1}{ \.{\mapsto} prevLogTerm ,\,}
\fl{ mentries}\al{291}{1}{ \.{\mapsto} entries ,\,}
\fl{}
\fl{}
\fl{ mlog}\al{294}{1}{ \.{\mapsto}}\al{294}{2}{ log [ i ] ,\,}
 \fl{ mcommitIndex}\al{295}{1}{ \.{\mapsto}}\al{295}{2}{ Min ( \{ commitIndex
 [ i ] ,\, lastEntry \} ) ,\,}
\fl{ msource}\al{296}{1}{ \.{\mapsto} i ,\,}
\fl{ mdest}\al{297}{1}{ \.{\mapsto} j ] )}
 \fl{ \.{\land}}\al{298}{1}{ {\UNCHANGED} {\langle} serverVars ,\,
 candidateVars ,\, leaderVars ,\, logVars {\rangle}}
\fl{}
\fl{}
\fl{}
\fl{}
\fl{ BecomeLeader ( i ) \.{\defeq}}
\fl{ \.{\land}}\al{304}{1}{ state [ i ] \.{=} Candidate}
\fl{ \.{\land}}\al{305}{1}{ votesGranted [ i ] \.{\in} Quorum}\al{305}{7}{%
\@y{\@s{0}%
 获取法定人数的�?�票
}%
}
 \fl{ \.{\land}}\al{306}{1}{ state \.{'}}\al{306}{3}{ \.{=}}\al{306}{4}{ [
 state {\EXCEPT} {\bang} [ i ] \.{=} Leader ]}\al{306}{14}{%
\@y{\@s{0}%
 变更状�?�为领导�?
}%
}
 \fl{ \.{\land}}\al{307}{1}{ nextIndex \.{'}}\al{307}{3}{ \.{=}}\al{307}{4}{
 [}\al{307}{5}{ nextIndex {\EXCEPT} {\bang} [ i ] \.{=}}\al{307}{12}{%
\@y{\@s{0}%
 设置\ensuremath{nextIndex},设置为自己的日志数\ensuremath{\.{+}1
}}%
}
\fl{ [ j \.{\in} Server \.{\mapsto} Len ( log [ i ] ) \.{+} 1 ] ]}
 \fl{ \.{\land}}\al{309}{1}{ matchIndex \.{'}}\al{309}{3}{ \.{=}
 [}\al{309}{5}{ matchIndex {\EXCEPT} {\bang} [ i ] \.{=}}\al{309}{12}{%
\@y{\@s{0}%
 将所有的\ensuremath{matchIndex}设置�?0,之后会进行沟�??
}%
}
\fl{ [ j \.{\in} Server \.{\mapsto} 0 ] ]}
 \fl{ \.{\land}}\al{311}{1}{ elections \.{'}}\al{311}{3}{ \.{=}}\al{311}{4}{
 elections \.{\cup}%
\@y{\@s{0}%
 将此次�?�举加入成功选举的集合中
}%
}
 \fl{ \{ [}\al{312}{2}{ eterm}\al{312}{3}{ \.{\mapsto}}\al{312}{4}{
 currentTerm [ i ] ,\,}
\fl{ eleader}\al{313}{1}{ \.{\mapsto}}\al{313}{2}{ i ,\,}
\fl{ elog}\al{314}{1}{ \.{\mapsto} log [ i ] ,\,}
\fl{ evotes}\al{315}{1}{ \.{\mapsto} votesGranted [ i ] ,\,}
\fl{ evoterLog}\al{316}{1}{ \.{\mapsto} voterLog [ i ] ] \}}
 \fl{ \.{\land}}\al{317}{1}{ {\UNCHANGED} {\langle} messages ,\, currentTerm
 ,\, votedFor ,\, candidateVars ,\, logVars {\rangle}}
\fl{}
\fl{}
\fl{}
\fl{}
\fl{}
\fl{}
\fl{ ClientRequest ( i ,\, v ) \.{\defeq}}
\fl{ \.{\land}}\al{325}{1}{ state [ i ] \.{=} Leader}\al{325}{7}{%
\@y{\@s{0}%
 只有领导者才可以接受客户端请�?
}%
}
 \fl{ \.{\land}}\al{326}{1}{ \.{\LET}}\al{326}{2}{ entry \.{\defeq}
 [}\al{326}{5}{ term}\al{326}{6}{ \.{\mapsto}}\al{326}{7}{ currentTerm [ i ]
 ,\,}\al{326}{12}{%
\@y{\@s{0}%
 追加日志
}%
}
\fl{ value}\al{327}{1}{ \.{\mapsto}}\al{327}{2}{ v ]}
\fl{ newLog \.{\defeq} Append ( log [ i ] ,\, entry )}
 \fl{ \.{\IN}}\al{329}{1}{ log \.{'} \.{=} [ log {\EXCEPT} {\bang} [ i ] \.{=}
 newLog ]}
 \fl{ \.{\land}}\al{330}{1}{ {\UNCHANGED} {\langle}}\al{330}{3}{ messages ,\,
 serverVars ,\, candidateVars ,\,}
\fl{ leaderVars ,\, commitIndex {\rangle}}
\fl{}
\fl{}
\fl{}
\fl{}
\fl{}
\fl{ AdvanceCommitIndex ( i ) \.{\defeq}}
\fl{ \.{\land}}\al{338}{1}{ state [ i ] \.{=} Leader}
\fl{ \.{\land}}\al{339}{1}{ \.{\LET}}\al{339}{2}{%
\@y{\@s{0}%
 The set of servers that agree up through index.
}%
}
 \fl{ Agree ( index ) \.{\defeq} \{ i \} \.{\cup} \{ k}\al{340}{11}{ \.{\in}
 Server \.{:}}
\fl{ matchIndex [ i ] [ k ] \.{\geq} index \}}
\fl{}
 \fl{ agreeIndexes \.{\defeq} \{}\al{343}{3}{ index \.{\in} 1 \.{\dotdot} Len
 ( log [ i ] ) \.{:}}
\fl{ Agree ( index ) \.{\in} Quorum \}}
\fl{}
\fl{ newCommitIndex \.{\defeq}}
\fl{ {\IF}}\al{347}{1}{ \.{\land}}\al{347}{2}{ agreeIndexes \.{\neq} \{ \}}
 \fl{ \.{\land}}\al{348}{1}{ log [ i ] [ Max ( agreeIndexes ) ] . term \.{=}
 currentTerm [ i ]}
\fl{ \.{\THEN}}
\fl{ Max ( agreeIndexes )}
\fl{ \.{\ELSE}}
\fl{ commitIndex [ i ]}
 \fl{ \.{\IN} commitIndex \.{'} \.{=} [ commitIndex {\EXCEPT} {\bang} [ i ]
 \.{=} newCommitIndex ]}
 \fl{ \.{\land}}\al{354}{1}{ {\UNCHANGED} {\langle} messages ,\, serverVars
 ,\, candidateVars ,\, leaderVars ,\, log {\rangle}}
\fl{}
\fl{}
\fl{}
\fl{}
\fl{}\midbar\cl{}
\fl{}
\fl{}
\fl{}
\fl{}
\fl{}
\fl{ HandleRequestVoteRequest ( i ,\, j ,\, m ) \.{\defeq}}
 \fl{ \.{\LET}}\al{366}{1}{ logOk}\al{366}{2}{ \.{\defeq}}\al{366}{3}{
 \.{\lor}}\al{366}{4}{ m . mlastLogTerm \.{>} LastTerm ( log [ i ] )}
 \fl{ \.{\lor}}\al{367}{1}{ \.{\land}}\al{367}{2}{ m . mlastLogTerm \.{=}
 LastTerm ( log [ i ] )}
\fl{ \.{\land}}\al{368}{1}{ m . mlastLogIndex \.{\geq} Len ( log [ i ] )}
 \fl{ grant}\al{369}{1}{ \.{\defeq}}\al{369}{2}{ \.{\land}}\al{369}{3}{ m .
 mterm \.{=} currentTerm [ i ]}
\fl{ \.{\land}}\al{370}{1}{ logOk}
\fl{ \.{\land}}\al{371}{1}{ votedFor [ i ] \.{\in} \{ Nil ,\, j \}}
 \fl{ \.{\IN}}\al{372}{1}{ \.{\land}}\al{372}{2}{ m . mterm \.{\leq}
 currentTerm [ i ]}
 \fl{ \.{\land}}\al{373}{1}{ \.{\lor}}\al{373}{2}{ grant}\al{373}{3}{
 \.{\land}}\al{373}{4}{ votedFor \.{'} \.{=} [ votedFor {\EXCEPT} {\bang} [ i
 ] \.{=} j ]}
 \fl{ \.{\lor}}\al{374}{1}{ {\lnot} grant}\al{374}{3}{ \.{\land}}\al{374}{4}{
 {\UNCHANGED} votedFor}
 \fl{ \.{\land}}\al{375}{1}{ Reply ( [}\al{375}{4}{ mtype}\al{375}{5}{
 \.{\mapsto}}\al{375}{6}{ RequestVoteResponse ,\,}
\fl{ mterm}\al{376}{1}{ \.{\mapsto}}\al{376}{2}{ currentTerm [ i ] ,\,}
\fl{ mvoteGranted}\al{377}{1}{ \.{\mapsto} grant ,\,}
\fl{}
\fl{}
\fl{ mlog}\al{380}{1}{ \.{\mapsto}}\al{380}{2}{ log [ i ] ,\,}
\fl{ msource}\al{381}{1}{ \.{\mapsto}}\al{381}{2}{ i ,\,}
\fl{ mdest}\al{382}{1}{ \.{\mapsto} j ] ,\,}
\fl{ m )}
 \fl{ \.{\land}}\al{384}{1}{ {\UNCHANGED} {\langle} state ,\, currentTerm ,\,
 candidateVars ,\, leaderVars ,\, logVars {\rangle}}
\fl{}
\fl{}
\fl{}
\fl{ HandleRequestVoteResponse ( i ,\, j ,\, m ) \.{\defeq}}
\fl{}
\fl{}
\fl{ \.{\land}}\al{391}{1}{ m . mterm \.{=} currentTerm [ i ]}
 \fl{ \.{\land}}\al{392}{1}{ votesResponded \.{'} \.{=} [}\al{392}{5}{
 votesResponded {\EXCEPT} {\bang} [ i ] \.{=}}
\fl{ votesResponded [ i ] \.{\cup} \{ j \} ]}
 \fl{ \.{\land}}\al{394}{1}{ \.{\lor}}\al{394}{2}{ \.{\land}}\al{394}{3}{ m .
 mvoteGranted}
 \fl{ \.{\land}}\al{395}{1}{ votesGranted \.{'} \.{=} [}\al{395}{5}{
 votesGranted {\EXCEPT} {\bang} [ i ] \.{=}}
\fl{ votesGranted [ i ] \.{\cup} \{ j \} ]}
 \fl{ \.{\land}}\al{397}{1}{ voterLog \.{'} \.{=} [}\al{397}{5}{ voterLog
 {\EXCEPT} {\bang} [ i ] \.{=}}
\fl{ voterLog [ i ] \.{\,@@\,} ( j \.{\colongt} m . mlog ) ]}
\fl{ \.{\lor}}\al{399}{1}{ \.{\land}}\al{399}{2}{ {\lnot} m . mvoteGranted}
 \fl{ \.{\land}}\al{400}{1}{ {\UNCHANGED} {\langle} votesGranted ,\, voterLog
 {\rangle}}
\fl{ \.{\land}}\al{401}{1}{ Discard ( m )}
 \fl{ \.{\land}}\al{402}{1}{ {\UNCHANGED} {\langle} serverVars ,\, votedFor
 ,\, leaderVars ,\, logVars {\rangle}}
\fl{}
\fl{}
\fl{}
\fl{}
\fl{}
\fl{ HandleAppendEntriesRequest ( i ,\, j ,\, m ) \.{\defeq}}
 \fl{ \.{\LET} logOk \.{\defeq}}\al{409}{3}{ \.{\lor}}\al{409}{4}{ m .
 mprevLogIndex \.{=} 0}
 \fl{ \.{\lor}}\al{410}{1}{ \.{\land}}\al{410}{2}{ m .
 mprevLogIndex}\al{410}{5}{ \.{>} 0}
 \fl{ \.{\land}}\al{411}{1}{ m . mprevLogIndex}\al{411}{4}{ \.{\leq} Len ( log
 [ i ] )}
 \fl{ \.{\land}}\al{412}{1}{ m . mprevLogTerm \.{=} log [ i ] [ m .
 mprevLogIndex ] . term}
 \fl{ \.{\IN}}\al{413}{1}{ \.{\land}}\al{413}{2}{ m . mterm \.{\leq}
 currentTerm [ i ]}
\fl{ \.{\land}}\al{414}{1}{ \.{\lor}}\al{414}{2}{ \.{\land}}\al{414}{3}{%
\@y{\@s{0}%
 reject request
}%
}
\fl{ \.{\lor}}\al{415}{1}{ m . mterm \.{<} currentTerm [ i ]}
 \fl{ \.{\lor}}\al{416}{1}{ \.{\land}}\al{416}{2}{ m . mterm \.{=} currentTerm
 [ i ]}
\fl{ \.{\land}}\al{417}{1}{ state [ i ] \.{=} Follower}
\fl{ \.{\land}}\al{418}{1}{ {\lnot} logOk}
 \fl{ \.{\land}}\al{419}{1}{ Reply ( [}\al{419}{4}{ mtype}\al{419}{5}{
 \.{\mapsto}}\al{419}{6}{ AppendEntriesResponse ,\,}
\fl{ mterm}\al{420}{1}{ \.{\mapsto}}\al{420}{2}{ currentTerm [ i ] ,\,}
\fl{ msuccess}\al{421}{1}{ \.{\mapsto} {\FALSE} ,\,}
\fl{ mmatchIndex}\al{422}{1}{ \.{\mapsto} 0 ,\,}
\fl{ msource}\al{423}{1}{ \.{\mapsto} i ,\,}
\fl{ mdest}\al{424}{1}{ \.{\mapsto} j ] ,\,}
\fl{ m )}
 \fl{ \.{\land}}\al{426}{1}{ {\UNCHANGED} {\langle} serverVars ,\, logVars
 {\rangle}}
\fl{ \.{\lor}}\al{427}{1}{%
\@y{\@s{0}%
 return to follower state
}%
}
\fl{ \.{\land}}\al{428}{1}{ m . mterm \.{=} currentTerm [ i ]}
\fl{ \.{\land}}\al{429}{1}{ state [ i ] \.{=} Candidate}
 \fl{ \.{\land}}\al{430}{1}{ state \.{'} \.{=} [ state {\EXCEPT} {\bang} [ i ]
 \.{=} Follower ]}
 \fl{ \.{\land}}\al{431}{1}{ {\UNCHANGED} {\langle} currentTerm ,\, votedFor
 ,\, logVars ,\, messages {\rangle}}
\fl{ \.{\lor}}\al{432}{1}{%
\@y{\@s{0}%
 accept request
}%
}
\fl{ \.{\land}}\al{433}{1}{ m . mterm \.{=} currentTerm [ i ]}
\fl{ \.{\land}}\al{434}{1}{ state [ i ] \.{=} Follower}
\fl{ \.{\land}}\al{435}{1}{ logOk}
 \fl{ \.{\land}}\al{436}{1}{ \.{\LET} index \.{\defeq} m . mprevLogIndex \.{+}
 1}
\fl{ \.{\IN}}\al{437}{1}{ \.{\lor}}\al{437}{2}{%
\@y{\@s{0}%
 already done with request
}%
}
 \fl{ \.{\land}}\al{438}{1}{ \.{\lor}}\al{438}{2}{ m . mentries \.{=}
 {\langle} {\rangle}}
 \fl{ \.{\lor}}\al{439}{1}{ \.{\land}}\al{439}{2}{ Len ( log [ i ] ) \.{\geq}
 index}
 \fl{ \.{\land}}\al{440}{1}{ log [ i ] [ index ] . term \.{=} m . mentries [ 1
 ] . term}
\fl{}
\fl{}
\fl{}
 \fl{ \.{\land}}\al{444}{1}{ commitIndex \.{'} \.{=} [}\al{444}{5}{
 commitIndex {\EXCEPT} {\bang} [ i ] \.{=}}
\fl{ m . mcommitIndex ]}
 \fl{ \.{\land}}\al{446}{1}{ Reply ( [}\al{446}{4}{ mtype}\al{446}{5}{
 \.{\mapsto}}\al{446}{6}{ AppendEntriesResponse ,\,}
\fl{ mterm}\al{447}{1}{ \.{\mapsto}}\al{447}{2}{ currentTerm [ i ] ,\,}
\fl{ msuccess}\al{448}{1}{ \.{\mapsto} {\TRUE} ,\,}
 \fl{ mmatchIndex}\al{449}{1}{ \.{\mapsto}}\al{449}{2}{ m . mprevLogIndex
 \.{+}}
\fl{ Len ( m . mentries ) ,\,}
\fl{ msource}\al{451}{1}{ \.{\mapsto} i ,\,}
\fl{ mdest}\al{452}{1}{ \.{\mapsto} j ] ,\,}
\fl{ m )}
 \fl{ \.{\land}}\al{454}{1}{ {\UNCHANGED} {\langle} serverVars ,\, logVars
 {\rangle}}
\fl{ \.{\lor}}\al{455}{1}{%
\@y{\@s{0}%
 conflict: remove 1 entry
}%
}
\fl{ \.{\land}}\al{456}{1}{ m . mentries \.{\neq} {\langle} {\rangle}}
\fl{ \.{\land}}\al{457}{1}{ Len ( log [ i ] ) \.{\geq} index}
 \fl{ \.{\land}}\al{458}{1}{ log [ i ] [ index ] . term}\al{458}{10}{ \.{\neq}
 m . mentries [ 1 ] . term}
 \fl{ \.{\land}}\al{459}{1}{ \.{\LET} new \.{\defeq} [}\al{459}{5}{
 index2}\al{459}{6}{ \.{\in} 1 \.{\dotdot} ( Len ( log [ i ] ) \.{-} 1 )
 \.{\mapsto}}
\fl{ log [ i ] [ index2 ] ]}
\fl{ \.{\IN} log \.{'} \.{=} [ log {\EXCEPT} {\bang} [ i ] \.{=} new ]}
 \fl{ \.{\land}}\al{462}{1}{ {\UNCHANGED} {\langle} serverVars ,\, commitIndex
 ,\, messages {\rangle}}
\fl{ \.{\lor}}\al{463}{1}{%
\@y{\@s{0}%
 no conflict: append entry
}%
}
\fl{ \.{\land}}\al{464}{1}{ m . mentries \.{\neq} {\langle} {\rangle}}
\fl{ \.{\land}}\al{465}{1}{ Len ( log [ i ] ) \.{=} m . mprevLogIndex}
 \fl{ \.{\land}}\al{466}{1}{ log \.{'} \.{=} [ log}\al{466}{6}{ {\EXCEPT}
 {\bang} [ i ] \.{=}}
\fl{ Append ( log [ i ] ,\, m . mentries [ 1 ] ) ]}
 \fl{ \.{\land}}\al{468}{1}{ {\UNCHANGED} {\langle} serverVars ,\, commitIndex
 ,\, messages {\rangle}}
 \fl{ \.{\land}}\al{469}{1}{ {\UNCHANGED} {\langle} candidateVars ,\,
 leaderVars {\rangle}}
\fl{}
\fl{}
\fl{}
\fl{ HandleAppendEntriesResponse ( i ,\, j ,\, m ) \.{\defeq}}
\fl{ \.{\land}}\al{474}{1}{ m . mterm \.{=} currentTerm [ i ]}
 \fl{ \.{\land}}\al{475}{1}{ \.{\lor}}\al{475}{2}{ \.{\land}}\al{475}{3}{ m .
 msuccess%
\@y{\@s{0}%
 successful
}%
}
 \fl{ \.{\land}}\al{476}{1}{ nextIndex \.{'}}\al{476}{3}{ \.{=}}\al{476}{4}{ [
 nextIndex {\EXCEPT} {\bang} [ i ] [ j ]}\al{476}{14}{ \.{=}}\al{476}{15}{ m
 . mmatchIndex \.{+} 1 ]}
 \fl{ \.{\land}}\al{477}{1}{ matchIndex \.{'}}\al{477}{3}{ \.{=}}\al{477}{4}{
 [ matchIndex {\EXCEPT} {\bang} [ i ] [ j ]}\al{477}{14}{ \.{=}}\al{477}{15}{
 m . mmatchIndex ]}
\fl{ \.{\lor}}\al{478}{1}{ \.{\land}}\al{478}{2}{ {\lnot} m . msuccess%
\@y{\@s{0}%
 not successful
}%
}
 \fl{ \.{\land}}\al{479}{1}{ nextIndex \.{'} \.{=} [}\al{479}{5}{ nextIndex
 {\EXCEPT} {\bang} [ i ] [ j ] \.{=}}
\fl{ Max ( \{ nextIndex [ i ] [ j ] \.{-} 1 ,\, 1 \} ) ]}
\fl{ \.{\land}}\al{481}{1}{ {\UNCHANGED} {\langle} matchIndex {\rangle}}
\fl{ \.{\land}}\al{482}{1}{ Discard ( m )}
 \fl{ \.{\land}}\al{483}{1}{ {\UNCHANGED} {\langle} serverVars ,\,
 candidateVars ,\, logVars ,\, elections {\rangle}}
\fl{}
\fl{}
\fl{ UpdateTerm ( i ,\, j ,\, m ) \.{\defeq}}
\fl{ \.{\land}}\al{487}{1}{ m . mterm \.{>} currentTerm [ i ]}
 \fl{ \.{\land}}\al{488}{1}{ currentTerm \.{'}}\al{488}{3}{ \.{=}}\al{488}{4}{
 [ currentTerm {\EXCEPT} {\bang} [ i ]}\al{488}{11}{ \.{=}}\al{488}{12}{ m .
 mterm ]}
 \fl{ \.{\land}}\al{489}{1}{ state \.{'}}\al{489}{3}{ \.{=}}\al{489}{4}{ [
 state {\EXCEPT} {\bang} [ i ]}\al{489}{11}{ \.{=}}\al{489}{12}{ Follower ]}
 \fl{ \.{\land}}\al{490}{1}{ votedFor \.{'}}\al{490}{3}{ \.{=} [ votedFor
 {\EXCEPT} {\bang} [ i ]}\al{490}{11}{ \.{=} Nil ]}
\fl{}
 \fl{ \.{\land}}\al{492}{1}{ {\UNCHANGED} {\langle} messages ,\, candidateVars
 ,\, leaderVars ,\, logVars {\rangle}}
\fl{}
\fl{}
\fl{ DropStaleResponse ( i ,\, j ,\, m ) \.{\defeq}}
\fl{ \.{\land}}\al{496}{1}{ m . mterm \.{<} currentTerm [ i ]}
\fl{ \.{\land}}\al{497}{1}{ Discard ( m )}
 \fl{ \.{\land}}\al{498}{1}{ {\UNCHANGED} {\langle} serverVars ,\,
 candidateVars ,\, leaderVars ,\, logVars {\rangle}}
\fl{}
\fl{}
\fl{ Receive ( m ) \.{\defeq}}
\fl{ \.{\LET}}\al{502}{1}{ i}\al{502}{2}{ \.{\defeq}}\al{502}{3}{ m . mdest}
\fl{ j}\al{503}{1}{ \.{\defeq}}\al{503}{2}{ m . msource}
\fl{ \.{\IN}}\al{504}{1}{%
\@y{\@s{0}%
 Any \ensuremath{RPC} with a newer term causes the recipient to advance
}%
}
\fl{}\al{505}{0}{}
\fl{ \.{\lor}}\al{506}{1}{ UpdateTerm ( i ,\, j ,\, m )}
 \fl{ \.{\lor}}\al{507}{1}{ \.{\land}}\al{507}{2}{ m . mtype \.{=}
 RequestVoteRequest}
\fl{ \.{\land}}\al{508}{1}{ HandleRequestVoteRequest ( i ,\, j ,\, m )}
 \fl{ \.{\lor}}\al{509}{1}{ \.{\land}}\al{509}{2}{ m . mtype \.{=}
 RequestVoteResponse}
 \fl{ \.{\land}}\al{510}{1}{ \.{\lor}}\al{510}{2}{ DropStaleResponse ( i ,\, j
 ,\, m )}
\fl{ \.{\lor}}\al{511}{1}{ HandleRequestVoteResponse ( i ,\, j ,\, m )}
 \fl{ \.{\lor}}\al{512}{1}{ \.{\land}}\al{512}{2}{ m . mtype \.{=}
 AppendEntriesRequest}
\fl{ \.{\land}}\al{513}{1}{ HandleAppendEntriesRequest ( i ,\, j ,\, m )}
 \fl{ \.{\lor}}\al{514}{1}{ \.{\land}}\al{514}{2}{ m . mtype \.{=}
 AppendEntriesResponse}
 \fl{ \.{\land}}\al{515}{1}{ \.{\lor}}\al{515}{2}{ DropStaleResponse ( i ,\, j
 ,\, m )}
\fl{ \.{\lor}}\al{516}{1}{ HandleAppendEntriesResponse ( i ,\, j ,\, m )}
\fl{}
\fl{}
\fl{}\midbar\cl{}
\fl{}
\fl{}
\fl{}
\fl{ DuplicateMessage ( m ) \.{\defeq}}
\fl{ \.{\land}}\al{524}{1}{ Send ( m )}
 \fl{ \.{\land}}\al{525}{1}{ {\UNCHANGED} {\langle} serverVars ,\,
 candidateVars ,\, leaderVars ,\, logVars {\rangle}}
\fl{}
\fl{}
\fl{ DropMessage ( m ) \.{\defeq}}
\fl{ \.{\land}}\al{529}{1}{ Discard ( m )}
 \fl{ \.{\land}}\al{530}{1}{ {\UNCHANGED} {\langle} serverVars ,\,
 candidateVars ,\, leaderVars ,\, logVars {\rangle}}
\fl{}
\fl{}\midbar\cl{}
\fl{}
\fl{}
\fl{}
\fl{}
\fl{}
\fl{}
\fl{}
\fl{}
\fl{}
\fl{}
 \fl{ Next \.{\defeq}}\al{543}{2}{ \.{\land}}\al{543}{3}{
 \.{\lor}}\al{543}{4}{ \E\, i}\al{543}{6}{ \.{\in}}\al{543}{7}{
 Server}\al{543}{8}{ \.{:}}\al{543}{9}{ Restart ( i )}
 \fl{ \.{\lor}}\al{544}{1}{ \E\, i}\al{544}{3}{ \.{\in}}\al{544}{4}{
 Server}\al{544}{5}{ \.{:}}\al{544}{6}{ Timeout ( i )}
 \fl{ \.{\lor}}\al{545}{1}{ \E\, i ,\, j \.{\in} Server \.{:} RequestVote ( i
 ,\, j )}
 \fl{ \.{\lor}}\al{546}{1}{ \E\, i}\al{546}{3}{ \.{\in}}\al{546}{4}{ Server
 \.{:} BecomeLeader ( i )}
 \fl{ \.{\lor}}\al{547}{1}{ \E\, i}\al{547}{3}{ \.{\in}}\al{547}{4}{ Server
 ,\, v \.{\in} Value \.{:} ClientRequest ( i ,\, v )}
 \fl{ \.{\lor}}\al{548}{1}{ \E\, i}\al{548}{3}{ \.{\in} Server \.{:}
 AdvanceCommitIndex ( i )}
 \fl{ \.{\lor}}\al{549}{1}{ \E\, i ,\, j \.{\in} Server \.{:} AppendEntries (
 i ,\, j )}
 \fl{ \.{\lor}}\al{550}{1}{ \E\, m}\al{550}{3}{ \.{\in}}\al{550}{4}{ {\DOMAIN}
 messages}\al{550}{6}{ \.{:}}\al{550}{7}{ Receive ( m )}
 \fl{ \.{\lor}}\al{551}{1}{ \E\, m}\al{551}{3}{ \.{\in}}\al{551}{4}{ {\DOMAIN}
 messages}\al{551}{6}{ \.{:}}\al{551}{7}{ DuplicateMessage ( m )}
 \fl{ \.{\lor}}\al{552}{1}{ \E\, m}\al{552}{3}{ \.{\in} {\DOMAIN}
 messages}\al{552}{6}{ \.{:} DropMessage ( m )}
\fl{}
 \fl{ \.{\land}}\al{554}{1}{ allLogs \.{'} \.{=} allLogs \.{\cup} \{ log [ i ]
 \.{:} i \.{\in} Server \}}
\fl{}
\fl{}
\fl{}
\fl{ Spec \.{\defeq} Init \.{\land} {\Box} [ Next ]_{ vars}}
\fl{}
\fl{}\bottombar\cl{}
\fl{}
\fl{}
\fl{}
\fl{}
\fl{}
\fl{}
\fl{}
\fl{}
\fl{}
\fl{}
\fl{}
\fl{}
\fl{}
\fl{}
\fl{}
\end{document}
